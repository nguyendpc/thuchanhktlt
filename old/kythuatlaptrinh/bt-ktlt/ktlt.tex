\documentclass[12pt,a4paper]{article}
\usepackage[utf8]{vietnam}
\usepackage{amsmath}
\usepackage{amsfonts}
\usepackage{amssymb}

\usepackage{hyperref}
\hypersetup{
    colorlinks=true,
    linkcolor=blue,
    filecolor=magenta,      
    urlcolor=cyan,
}
\usepackage{listings}
\usepackage{xcolor}

\definecolor{codegreen}{rgb}{0,0.6,0}
\definecolor{codegray}{rgb}{0.5,0.5,0.5}
\definecolor{codepurple}{rgb}{0.58,0,0.82}
\definecolor{backcolour}{rgb}{0.95,0.95,0.92}
\lstdefinestyle{mystyle}{
    backgroundcolor=\color{backcolour},   
    commentstyle=\color{codegreen},
    keywordstyle=\color{magenta},
    numberstyle=\tiny\color{codegray},
    stringstyle=\color{codepurple},
    basicstyle=\ttfamily\footnotesize,
    breakatwhitespace=false,         
    breaklines=true,                 
    captionpos=b,                    
    keepspaces=true,                 
    numbers=left,                    
    numbersep=5pt,                  
    showspaces=false,                
    showstringspaces=false,
    showtabs=false,                  
    tabsize=2
}

\lstset{style=mystyle}

\begin{document}
\section{Đệ quy - Recursion}
\begin{enumerate}
	\item Nhập vào mảng số nguyên có ít nhất 10 phần tử, in ra:
	\begin{itemize}
		\item Các số là ước số của 12
		\item Các số lẻ có trong mảng
		\item Số lớn nhất, bé nhất trong dãy số
	\end{itemize}		
	 
	\item Tính $S(n) = 1 + 2 + 3 + ... + n - 1 + n$
	\item Tính $S(n) = \frac{1}{2} + \frac{2}{3} + \frac{3}{4} + \cdots + \frac{n}{n + 1}$
	\item Tính $T(n) = 1\times 2\times 3\times \cdots \times n$
	\item Tính $T(x,n) = x^n$
	\item Tìm UCLN và BCNN của 2 số a và b
	\item Tính tổng các giai thừa: $1! + 2! + 3!+ \cdots + N!$
	\item Đếm số lượng chữ số của số nguyên dương n
	\item Tính tổng các chữ số của số nguyên dương n
	\item Tính tích các chữ số của số nguyên dương n
	\item Tìm chữ số lớn nhất, nhỏ nhất của số nguyên dương n
	\item Kiểm tra số nguyên dương n có toàn chữ số lẻ hay không ?
	\item Kiểm tra số nguyên dương n có toàn chữ số chẵn hay không ?
	\item In ra hoán vị n số tự nhiên đầu tiên.
	
	\item Nhập vào bảng dữ liệu
		\begin{center}
			\begin{tabular}{ |c|c|c| } 
				\hline
					ID & Ten & RefID \\
				\hline
					1 & Con trung & 4 \\
				\hline
					2 & Cao Cao & 1 \\
				\hline	
					3 & Dong vat & 0 \\
				\hline
					4 & Thuc vat & 0 \\
				\hline
			\end{tabular}
		\end{center}
		Quy tắc: nếu $RefID = 0$ thì menu là menu cha, nếu $RefID = ID$ của dòng khác thì là menu con của menu có ID tương ứng. In ra:
		\begin{lstlisting}[language=bash]
			Dong vat, Con trung, Cao Cao,
			Thuc vat, 
		\end{lstlisting}
		
%	\item (*) Thuật toán dò đường (Mê cung)
\end{enumerate}


\section{Thử sai quay lui - Vét cạn}
\begin{enumerate}
	\item Cho dãy A gồm n phần tử,
	\begin{enumerate}
		\item In ra tất cả các chỉnh hợp lặp chặp n của A.
		\item In ra tất cả các chỉnh hợp lặp chập n có tổng lớn hơn x của A.
	\end{enumerate}
	\item Đọc và ghi file. Đọc mảng. Đọc ma trận.
	\item Đọc dãy A từ file data.txt,
	\begin{enumerate}
		\item Ghi tất cả các chỉnh hợp lặp chặp n của A vào file result.txt.
		\item Ghi tất cả các chỉnh hợp lặp chập n có tổng lớn hơn x của A vào file result.txt.
	\end{enumerate}
	\item Cho dãy A gồm n phần tử, in ra tất cả các tổ hợp chập k của A.
	\item Cho dãy A gồm n phần tử, in ra tất cả các tập hợp con của A.
	\item Hãy liệt kê tất các chuỗi nhị phân có chiều dài là n theo thứ tự từ điển.
\textbf{Dữ liệu nhập:}\\
- Là số nguyên n ($1 \leq n \leq 20$)\\
\textbf{Dữ liệu xuất:}\\
- Các chuỗi nhị phân chiều dài n theo thứ tự từ điển.
	\item Một người mẹ muốn chia đều cho hai đứa con số tiền trong túi của mình. Khi rút tiền trong túi ra, bà thấy có n đồng tiền ($n \leq 20$) với các loại mệnh giá khác nhau. Bà cảm thấy lo lắng chia cho hai người con có được không. \\
\textbf{Yêu cầu:} Bạn hãy giúp bà ấy chia tiền cho hai người con. Nếu chia được thì nêu rõ số cách chia. \\
\textbf{Dữ liệu vào:} Gồm 02 dòng:
	\begin{enumerate}
		\item Dòng đầu ghi số n là số đồng tiền ($1 \leq n \leq 20$)
		\item Dòng sau ghi mệnh giá các đồng tiền là các số nguyên dương có cùng đơn vị tính ($0 \leq t\leq 500$).
	\end{enumerate}
\textbf{Dữ liệu ra:}\\
Dòng thứ nhất ghi số cách chia, nếu không thể chia được thì ghi “ Khong chia duoc”. Các dòng sau trong trường hợp chia được, mỗi dòng là 01 cách chia với quy ước người thứ nhất có tên là “A”, người thứ hai có tên là “B” và phải tương ứng với thứ tự của các đồng tiền mà bà đưa ra. 
	\item Cho dãy A gồm n phần tử, in ra tất cả các hoán vị của A.
	
	\item Cho dãy A gồm n phần tử, in ra tất cả các chỉnh hợp của A.
	\item OCSE - Ốc sên ăn rau\\
	\url{http://ntucoder.net/Problem/Details/51}
	\item Có n quả cân có trọng lượng lần lượt là $a_1, a_2, a_3,\cdots, a_n$. Hãy trình bày cách bố trí các quả cân lên 2 đĩa cân sao cho cân thăng bằng.
\end{enumerate}

\section{Nhánh cận}
\begin{enumerate}
	\item Cho dãy A gồm n phần tử, In ra tất cả các dãy con có tổng nhỏ hơn x của A.
	\item Bài toán ô số Puzzle: Cho một bảng $3 \times 3$ với 8 ô (mỗi ô có một số từ 1
đến 8) và một khoảng trống. Mục tiêu là di chuyển các số trên các ô để
khớp với cấu hình cuối cùng (sử dụng khoảng trống để dịch chuyển vị trí) .
Chúng ta có thể trượt 1 số đến bốn ô liền kề (trái, phải, trên và dưới) vào
khoảng trống.\\
\url{https://www.geeksforgeeks.org/8-puzzle-problem-using-branch-and-bound/}
	\item
	Có thể định nghĩa khái niệm dãy ngoặc đúng dưới dạng đệ quy như sau:\\
1. () là dãy ngoặc đúng\\
2. C là dãy ngoặc đúng nếu C = (A) hay C = AB với A, B là các dãy ngoặc đúng.\\
Ví dụ dãy ngoặc đúng: (), (()), ()(), (())()\\
Ví dụ dãy ngoặc sai: )(, ((((, ()((, )))), )()(\\
Bạn hãy viết chương trình liệt kê tất cả các dãy ngoặc đúng có chiều dài n (n chẵn)\\
\textbf{Dữ liệu nhập:}\\
- Là số nguyên n (n chẵn, $2 \leq n \leq 20$)\\
\textbf{Dữ liệu xuất:} với m là số lượng các dãy ngoặc đúng có chiều dài n
- Trong m dòng đầu tiên, mỗi dòng liệt kê một dãy ngoặc đúng chiều dài n. Các dãy được liệt kê theo thứ tự từ điển: '(' < ')'.\\
- Dòng cuối cùng: là số m \\
\url{http://ntucoder.net/Problem/Details/139}
	\item Có 1 robot mỗi bước có thể bước được 1m hoặc 2m. Hỏi đoạn đường n mét có bao nhiêu cách bước? In ra cách bước của Robot.
\end{enumerate}

\section{Chia để trị}
\begin{enumerate}
	\item Tính lũy thừa $a^n$
	\item Cho 1 dãy đã được sắp $x_1,x_2,\cdots,x_n$ và 1 số X. Tìm vị trí phần tử mang giá trị X thuộc dãy, nếu không có trả về giá trị -1.
	\item Cho mảng A chưa sắp xếp và một số nguyên X, đếm số lần xuất hiện của X trong mảng A bằng phương pháp chia để trị.
	
	\item Cho một khoảng [L,R] (L < R), hãy tạo ngẫu nhiên một hoán vị từ khoảng đó.
	

	\item Sàn trong mảng được sắp. Cho 1 mảng đã được sắp và 1 giá trị X, sàn của X là phần tử lớn nhất bé hơn bằng X thuộc dãy.
	\item Cho 1 mảng chứa các ký số 0,1. Các ký số 1 nằm đầu dãy, ký số 0 nằm cuối dãy. Đếm số lượng số 0 thuộc dãy.
	\item Tìm điểm cố định. Điểm cố định của mảng a[] là chỉ số i, sao cho a[i]=i; Cho 1 mảng, tìm điểm cố định của mảng.
	
\end{enumerate}

\section{Quy hoạch động}
\begin{enumerate}
	\item In ra dãy fibonacci n.
	\item Cho một dãy n số nguyên $a_1, a_2, a_3,\cdots,a_n$. Hãy tìm hai chỉ số i, j sao cho i < j và hiệu $a_j - a_i$ là lớn nhất.
	\item Cho dãy A gồm n phần tử, 
	\begin{enumerate}
		\item In tất cả các dãy con tăng.
		\item In tất cả các dãy con tăng lớn nhất.
		\item In tất cả các dãy con tăng dài nhất.
		\item In tất cả các dãy con tăng dài nhất và lớn nhất.
	\end{enumerate}
	\item Cho dãy A gồm n phần tử, In ra tất cả các dãy con có tổng lớn nhất của A.
\end{enumerate}

\section{Thuật toán tham lam}
\begin{enumerate}
	\item 
	  Trong nhà Nam hiện đang có n ổ cắm điện rời. Số lượng chỗ cắm trên mỗi ổ cắm điện này lần lượt là $a_1, a_2, a_3,\cdots,a_n$  chỗ cắm. Trên tường nhà Nam có một chỗ cắm cố định đang có điện. Vậy để cho một ổ cắm điện rời có điện thì phải cắm ổ cắm đó vào chỗ cắm cố định trên tường. Chúng ta cũng có thể cắm ổ cắm điện rời này vào một ổ cắm điện rời khác đang có điện.\\
      Nam có m thiết bị sử dụng điện, để sử dụng thì các thiết bị này cần được cắm vào ổ cắm trên tường hoặc ổ cắm rời đang có điện. Bạn hãy giúp Nam tìm ra số ổ cắm rời ít nhất cần dùng để có thể sử dụng tất cả m thiết bị điện này.

\textbf{Dữ liệu vào:} \\
      - Dòng thứ nhất gồm 2 số nguyên n, m cách nhau một khoảng trắng, dữ liệu vào đảm bảo $1\leq n,m \leq 50$, n là số lượng ổ cắm và m là số lượng thiết bị.\\
      - Dòng thứ hai gồm n số nguyên $a_1, a_2, a_3,\cdots,a_n$ là số chỗ cắm trên các ổ cắm rời tương ứng, mỗi số cách nhau một khoảng trắng, dữ liệu vào đảm bảo $1 \leq a_i \leq 50$.\\

\textbf{Dữ liệu ra:} là số nguyên cho biết số ổ cắm rời ít nhất cần sử dụng là bao nhiêu. Nếu đã sử dụng hết tất cả ổ cắm rời mà vẫn không đủ, in ra -1. \\
\url{http://ntucoder.net/Problem/Details/4}

	\item Cho n công việc, công việc i hoàn thành trong thời gian $t_i$, các công việc được thực hiện trên các máy công suất như nhau, mỗi máy đều có thể được công việc bất kỳ trong n công việc) mỗi công việc làm liên tục trên 1 máy cho đến khi hoàn thành. Hãy dùng ít máy nhất thực hiện đủ n công việc sao cho thời gian thực hiện hoàn thành n công việc trong khoảng thời gian cho trước là $T_0$.
	\item (Ghi đĩa CD) Cho n bài hát. Bài hát thứ i có dung lương $h_i$, Mỗi đĩa CD có dung lượng là M. Hỏi cần tối thiểu bao nhiêu đĩa để ghi tất cả các bài hát. (Mỗi bài hát ghi trọn ven trong 1 đĩa).	
	\item Fitting Shelves Problem \\
	https://www.geeksforgeeks.org/fitting-shelves-problem/
	\item Bài toán trồng cây \\
	Một nông dân đang muốn trồng hoa vào khu vườn của mình. Để cho khu vườn trở nên thật màu sắc ông quyết định trồng nhiều loài hoa khác nhau vào khu vườn. Mỗi loài hoa có 1 cách trồng khác nhau do đó ông sẽ trồng từng loài hoa vào các ngày liên tiếp nhau. Cháu của ông rất mong chờ được thấy tất cả loài hoa trong khu vườn đều nở hoa trông sẽ tuyệt vời như thế nào. Tuy nhiên mỗi loài hoa lại có thời gian phát triển từ lúc trồng tới lúc nở hoa khác nhau. Nhiệm vụ của bạn là giúp ông nông dân tìm ra ngày sớm nhất mà tất cả loài hoa đều nở hoa.\\
\begin{center}
	\begin{tabular}{|c|c|c|}
		\hline
		& Loài hoa & Thời gian nở \\
		\hline
		1 & Hoa hồng & 3 \\
		\hline
		2 & Hoa lan & 4 \\
		\hline
		3 & Hoa cúc & 2 \\
		\hline
		4 & Hoa mười giờ & 1 \\
		\hline
	\end{tabular}
\end{center}

\end{enumerate}

\section{Sắp xếp - Sort}
\begin{enumerate}
	\item Selection Sort
	\item Bubble Sort
	\item Insertion Sort
	\item Merge Sort
	\item Heap Sort
	\item Quick Sort
	\item Radix Sort
	\item Bucket Sort
	\item Counting Sort
	\item ShellSort
\end{enumerate}

\section{Bài toán kinh điển}
\begin{enumerate}
	\item Tháp Hà Nội
	\item 8 hậu
	\item Mã đi tuần
	\item Bài toán cái túi
	\item Bài toán người du lịch
	\item Cho 1 dãy a[n] chưa sắp xếp. Tìm phần tử nhỏ thứ k.
\end{enumerate}


\end{document}