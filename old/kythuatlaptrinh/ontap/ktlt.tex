\documentclass[12pt,a4paper]{article}
\usepackage[utf8]{vietnam}
\usepackage{amsmath}
\usepackage{amsfonts}
\usepackage{amssymb}


\usepackage{listings}
\usepackage{xcolor}

\definecolor{codegreen}{rgb}{0,0.6,0}
\definecolor{codegray}{rgb}{0.5,0.5,0.5}
\definecolor{codepurple}{rgb}{0.58,0,0.82}
\definecolor{backcolour}{rgb}{0.95,0.95,0.92}
\lstdefinestyle{mystyle}{
    backgroundcolor=\color{backcolour},   
    commentstyle=\color{codegreen},
    keywordstyle=\color{magenta},
    numberstyle=\tiny\color{codegray},
    stringstyle=\color{codepurple},
    basicstyle=\ttfamily\footnotesize,
    breakatwhitespace=false,         
    breaklines=true,                 
    captionpos=b,                    
    keepspaces=true,                 
    numbers=left,                    
    numbersep=5pt,                  
    showspaces=false,                
    showstringspaces=false,
    showtabs=false,                  
    tabsize=2
}

\lstset{style=mystyle}
\begin{document}
\section{Ôn tập}
\subsection{Kỹ thuật tổ chức chương trình}
\begin{itemize}
	\item
	Phân loại ngôn ngữ: Declarative $\&$ imperative language
	\item
	 Chuyển đổi chương trình:  Biên dịch (Compiler) và thông dịch (interpreter)
	 \item
	 Tối ưu hóa mã lệnh\\
	 Ví dụ: Tìm ước số chung lớn nhất của 2 số tự nhiên a,b


 Giải thuật 1 \\
\lstset{language=C++,keywordstyle=\color{blue}\ttfamily,
                stringstyle=\color{red}\ttfamily,
                commentstyle=\color{green}\ttfamily}
\begin{lstlisting}
int UCLN(int a, int b)
{
	while(a!=b)
	{
		if(a>b)
			a-=b;
		else
			b-=a;
	}
	return b;
}
\end{lstlisting}

Giải thuật 2  \\
\lstset{language=C++,keywordstyle=\color{blue}\ttfamily,
                stringstyle=\color{red}\ttfamily,
                commentstyle=\color{green}\ttfamily}
\begin{lstlisting}
int UCLN(int x, int y)
{
	int t = x % y;
	while (t!=0)
	{
		x=y;
		y=t;
		t=x % y;
	}
	return y;
}
\end{lstlisting}
	\item
	Phong cách lập trình: "Elements of Programming Style" - Kernighan $\&$ Plauger
\end{itemize}

\subsection{Ngôn ngữ lập trình C/C++}
\begin{itemize}
	\item
	Khai báo thư viện, khai báo biến, kiểu biến, mảng 1 chiều, mảng đa chiều,...
	\item
	Toán tử: toán tử số học, toán tử quan hệ, toán tử logic, toán tử so sánh, toán tử gán, toán tử hỗn hợp, thứ tự ưu tiên toán tử,...
	\item
	Cấu trúc điều khiển: Cấu trúc rẽ nhánh có điều kiện, Cấu trúc vòng lặp,...
	\item
	Các khái niệm: tham số, tham chiếu, tham trị, đối số, biến toàn cục, biến cục bộ.
	\item
	Macro, Hàm Inline
\end{itemize}
\subsection{Tham chiếu, tham trị}
\begin{itemize}
	\item
	Giá trị truyền cho tham chiếu là một biến; giá trị truyền cho tham trị có thể là 1
biểu thức
	\item
	Những thay đổi trong chương trình con có liên quan đến tham biến giữ lại, những
thay đổi trong chương trình con liên quan đến tham trị không ảnh hưởng đến đối
tượng truyền.
\end{itemize}

\lstset{language=C++,keywordstyle=\color{blue}\ttfamily,
                stringstyle=\color{red}\ttfamily,
                commentstyle=\color{green}\ttfamily}
\begin{lstlisting}
#include <stdio.h>
void Swap(int &a, int b){
	int tmp = a;
	a = b;
	b = tmp;
}
int main() {
	int first =5;
	int second =7;
	printf("\nfirst = %d, second = %d", first, second);
	Swap(first, second);
	printf("\nfirst = %d, second = %d", first, second);
}
\end{lstlisting}
\subsection{Đệ qui}
Khái niệm: Một khái niệm, định nghĩa được gọi là đệ qui nếu trong khái niệm/định nghĩa có chứa lại chính nó.\\
Một khái niệm đề qui gồm 2 thành phần:
\begin{itemize}
	\item
	Thành phần neo (dừng) (anchor): đảm bảo cho tính dừng
	\item
	Thành phần đề qui: thể hiện tính “quy nạp”
\end{itemize}
Phân loại:
\begin{itemize}
	\item Tuyến tính: Trong thân hàm có duy nhất một lời gọi hàm gọi lại chính nó một cách tường minh.
	\item Nhị phân: Trong thân hàm có hai lời gọi hàm gọi lại chính nó một cách tường minh.
	\item Hỗ tương: Trong thân hàm này có lời gọi hàm tới hàm kia và bên trong thân hàm kia có lời gọi hàm tới hàm này.
	\item Phi tuyến: Trong thân hàm có lời gọi hàm lại chính nó được đặt bên trong thân vòng lặp.
\end{itemize}
Các vấn đề đệ quy thông dụng
\begin{itemize}
	\item Truy hồi
	\item Chia để trị
	\item Lần ngược
\end{itemize}
Các bài tập đệ quy nổi tiếng
\begin{itemize}
	\item 8 hậu
	\item Mã đi tuần
	\item Tháp Hà nội
	\item Phát sinh hoán vị
\end{itemize}
\vspace{5cm}
\subsection{Đọc ghi file}
Ghi file
\lstset{language=C++,keywordstyle=\color{blue}\ttfamily,
                stringstyle=\color{red}\ttfamily,
                commentstyle=\color{green}\ttfamily}
\begin{lstlisting}
#include <stdio.h>

main() {
   FILE *fp;

   // mo file
   // fopen( "ten duong dan" , "che do" );
   fp = fopen("/tmp/test.txt", "w+");
   // in bang lenh fprintf
   fprintf(fp, "This is testing for fprintf...\n");
   // in bang lenh fputs
   fputs("This is testing for fputs...\n", fp);
   
   // dong file
   fclose(fp);
}
\end{lstlisting}
Đọc file
\lstset{language=C++,keywordstyle=\color{blue}\ttfamily,
                stringstyle=\color{red}\ttfamily,
                commentstyle=\color{green}\ttfamily}
\begin{lstlisting}
#include <stdio.h>

main() {

   FILE *fp;
   char buff[255];
   // mo file
   // fopen( "ten duong dan" , "che do" );
   fp = fopen("/tmp/test.txt", "r");
   
   // doc tung phan tu cua file
   fscanf(fp, "%s", buff);
   printf("1 : %s\n", buff );

   // doc tung doc cua file
   fgets(buff, 255, (FILE*)fp);
   printf("2: %s\n", buff );
   
   fgets(buff, 255, (FILE*)fp);
   printf("3: %s\n", buff );
   
   // dong file
   fclose(fp);
}
\end{lstlisting}

\begin{tabular}{|p{2cm}|p{5cm}|p{6cm}|}
\hline
\textbf{Mode} &	\textbf{Ý nghĩa}	& \textbf{Nếu file không tồn tại}\\
\hline
r &	Mở file chỉ cho phép đọc &	Nếu file không tồn tại, fopen() trả về NULL.\\
\hline
rb &	Mở file chỉ cho phép đọc dưới dạng nhị phân. &	Nếu file không tồn tại, fopen() trả về NULL.\\
\hline
w &	Mở file chỉ cho phép ghi.&	Nếu file đã tồn tại, nội dung sẽ bị ghi đè.  Nếu file không tồn tại, nó sẽ được tạo tự động.\\
\hline
wb &	Open for writing in binary mode.	& Nếu file đã tồn tại, nội dung sẽ bị ghi đè.  Nếu file không tồn tại, nó sẽ được tạo tự động.\\
\hline
a &	Mở file ở chế độ ghi “append”. Tức là sẽ ghi vào cuối của nội dung đã có.	& Nếu file không tồn tại, nó sẽ được tạo tự động.\\
\hline
ab &	Mở file ở chế độ ghi nhị phân “append”. Tức là sẽ ghi vào cuối của nội dung đã có. &	Nếu file không tồn tại, nó sẽ được tạo tự động.\\
\hline
r+ &	Mở file cho phép cả đọc và ghi. &	Nếu file không tồn tại, fopen() trả về NULL.\\
\hline
rb+	& Mở file cho phép cả đọc và ghi ở dạng nhị phân. &	Nếu file không tồn tại, fopen() trả về NULL.\\
\hline
w+	& Mở file cho phép cả đọc và ghi. &	Nếu file đã tồn tại, nội dung sẽ bị ghi đè. Nếu file không tồn tại, nó sẽ được tạo tự động.\\
\hline
wb+	& Mở file cho phép cả đọc và ghi ở dạng nhị phân. &	Nếu file đã tồn tại, nội dung sẽ bị ghi đè. Nếu file không tồn tại, nó sẽ được tạo tự động.\\
\hline
a+	& Mở file cho phép đọc và ghi “append”.	& Nếu file không tồn tại, nó sẽ được tạo tự động.\\
\hline
ab+	& Mở file cho phép đọc và ghi “append” ở dạng nhị phân.	& Nếu file không tồn tại, nó sẽ được tạo tự động.\\
\hline
\end{tabular}



\section{Bài tập}
\subsection{Đệ quy - Recursion}
\begin{enumerate}
	\item Nhập vào mảng số nguyên có ít nhất 10 phần tử, in ra:
	\begin{itemize}
		\item Các số là ước số của 12
		\item Các số lẻ có trong mảng
		\item Số lớn nhất, bé nhất trong dãy số
	\end{itemize}		
	 
	\item Tính $S(n) = 1 + 2 + 3 + ... + n - 1 + n$
	\item Tính $S(n) = \frac{1}{2} + \frac{2}{3} + \frac{3}{4} + \cdots + \frac{n}{n + 1}$
	\item Tính $T(n) = 1\times 2\times 3\times \cdots \times n$
	\item Tính $T(x,n) = x^n$
	\item Tìm UCLN và BCNN của 2 số a và b
	\item Tính tổng các giai thừa: $1! + 2! + 3!+ \cdots + N!$
	\item Đếm số lượng chữ số của số nguyên dương n
	\item Tính tổng các chữ số của số nguyên dương n
	\item Tính tích các chữ số của số nguyên dương n
	\item Tìm chữ số lớn nhất, nhỏ nhất của số nguyên dương n
	\item Kiểm tra số nguyên dương n có toàn chữ số lẻ hay không ?
	\item Kiểm tra số nguyên dương n có toàn chữ số chẵn hay không ?
	\item In ra hoán vị n số tự nhiên đầu tiên.
	
	\item Nhập vào bảng dữ liệu
		\begin{center}
			\begin{tabular}{ |c|c|c| } 
				\hline
					ID & Ten & RefID \\
				\hline
					1 & Con trung & 4 \\
				\hline
					2 & Cao Cao & 1 \\
				\hline	
					3 & Dong vat & 0 \\
				\hline
					4 & Thuc vat & 0 \\
				\hline
			\end{tabular}
		\end{center}
		Quy tắc: nếu $RefID = 0$ thì menu là menu cha, nếu $RefID = ID$ của dòng khác thì là menu con của menu có ID tương ứng. In ra:
		\begin{lstlisting}[language=bash]
			Dong vat, Con trung, Cao Cao,
			Thuc vat, 
		\end{lstlisting}
		
%	\item (*) Thuật toán dò đường (Mê cung)
\end{enumerate}
\subsection{Quay lui, Nhánh cận - Backtracking, Branch and Bound}
\begin{enumerate}
	\item Có 1 robot mỗi bước có thể bước được 1m hoặc 2m. Hỏi đoạn đường n mét có bao nhiêu cách bước? In ra cách bước của Robot.
	\item Có n quả cân có trọng lượng lần lượt là $a_1, a_2, a_3,\cdots, a_n$. Hãy trình bày cách bố trí các quả cân lên 2 đĩa cân sao cho cân thăng bằng.
	\item Tìm tất cả các cặp số nguyên x,y sao cho $x + y < 10$
\end{enumerate}
\subsection{Chia để trị - Divide Conquer}
\begin{enumerate}
	\item (*) Cho một tập hợp n điểm trong không gian d chiều, xác định 2 điểm có khoảng cách Euclide nhỏ nhất \\
	https://www.geeksforgeeks.org/closest-pair-of-points-using-divide-and-conquer-algorithm/?ref=rp
	\item Cho 1 mảng đã được sắp và 1 giá trị X, sàn của X là phần tử lớn nhất bé hơn bằng X thuộc dãy.
	\item Cho 1 mảng chứa các ký số 0,1. Các ký số 1 nằm đầu dãy, ký số 0 nằm cuối dãy. Đếm số lượng số 0 thuộc dãy
	\item Điểm cố định của mảng a[] là chỉ số i, sao cho $a[i]=i$. Cho 1 mảng, tìm điểm cố định của mảng.
	\item Cho một tập hợp các chuỗi, tìm tiền tố chung dài nhất. \\
	https://www.geeksforgeeks.org/longest-common-prefix-using-divide-and-conquer-algorithm/?ref=rp
	\item Cho mảng số nguyên arr[], tìm mảng con có tổng lớn nhất. \\
	https://www.geeksforgeeks.org/maximum-sum-subarray-using-divide-and-conquer-set-2/?ref=rp
\end{enumerate}
\subsection{Thuật toán tham lam - Greed Algorithm}
\begin{enumerate}
%	\item Bài toán người du lịch (TSP – Travelling Salesman Problem)\\
%	Một nguời du lịch muốn đi tham quan n thành phố $T_1, T_2,\cdots,T_n$ . Xuất phát từ một thành phố nào đó, người du lịch muốn đi qua tất cả các thành phố còn lại, mỗi thành phố đi qua duy nhất 1 lần rối quay trở lại thành phố xuất phát. Gọi $C_{ij}$ là chi phí đi từ thành phố $T_i$ đến $T_j$. Hãy tìm một hành trình thỏa yêu cầu bài toán sao cho chi phí là nhỏ nhất.
%	\item Bài toán xếp lịch: Cho n công việc, công việc i hoàn thành trong thời gian $t_i$, các công việc được thực hiện trên M máy công suất như nhau, mỗi máy đều có thể được công việc bất kỳ trong n công việc) mỗi công việc làm liên tục trên 1 máy cho đến khi hoàn thành. Hãy tổ chức máy thực hiện đủ n công việc sao cho thời gian thực hiện càng nhỏ càng tốt.
	\item Cho n công việc, công việc i hoàn thành trong thời gian $t_i$, các công việc được thực hiện trên các máy công suất như nhau, mỗi máy đều có thể được công việc bất kỳ trong n công việc) mỗi công việc làm liên tục trên 1 máy cho đến khi hoàn thành. Hãy dùng ít máy nhất thực hiện đủ n công việc sao cho thời gian thực hiện hoàn thành n công việc trong khoảng thời gian cho trước là $T_0$.
	\item (Ghi đĩa CD) Cho n bài hát. Bài hát thứ i có dung lương $h_i$, Mỗi đĩa CD có dung lượng là M. Hỏi cần tối thiểu bao nhiêu đĩa để ghi tất cả các bài hát. (Mỗi bài hát ghi trọn ven trong 1 đĩa).	
	\item Fitting Shelves Problem \\
	https://www.geeksforgeeks.org/fitting-shelves-problem/
	\item Bài toán trồng cây \\
	Một nông dân đang muốn trồng hoa vào khu vườn của mình. Để cho khu vườn trở nên thật màu sắc ông quyết định trồng nhiều loài hoa khác nhau vào khu vườn. Mỗi loài hoa có 1 cách trồng khác nhau do đó ông sẽ trồng từng loài hoa vào các ngày liên tiếp nhau. Cháu của ông rất mong chờ được thấy tất cả loài hoa trong khu vườn đều nở hoa trông sẽ tuyệt vời như thế nào. Tuy nhiên mỗi loài hoa lại có thời gian phát triển từ lúc trồng tới lúc nở hoa khác nhau. Nhiệm vụ của bạn là giúp ông nông dân tìm ra ngày sớm nhất mà tất cả loài hoa đều nở hoa.\\
\begin{center}
	\begin{tabular}{|c|c|c|}
		\hline
		& Loài hoa & Thời gian nở \\
		\hline
		1 & Hoa hồng & 3 \\
		\hline
		2 & Hoa lan & 4 \\
		\hline
		3 & Hoa cúc & 2 \\
		\hline
		4 & Hoa mười giờ & 1 \\
		\hline
	\end{tabular}
\end{center}
\end{enumerate}
	
	
\end{document}